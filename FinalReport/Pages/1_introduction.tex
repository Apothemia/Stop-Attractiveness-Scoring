\section{Introduction}
\label{sec:introduction}

\fontsize{12}{20}\selectfont

Public transport is an essential component of daily life, which is why continuous efforts are made to make it more sustainable, accessible, and affordable.
Despite these efforts, many cities struggle with people preferring private to public transportation, causing variety of problems.

Recent advancements in data collection, especially with Automatic Fare Collection (AFC) systems and Smart Card Data (SCD), have been a key factor to analyse and understand passenger behaviour to optimise public transportation.
Smart card transactions reveal when and where passengers board or alight buses, offering insights into travel patterns and network dynamics.

This study aims to develop a \textbf{Bus Stop Attractiveness Scoring (BSAS)} framework using smart card data, encompassing multiple bus lines.
Unlike optimisation-based works that focus on improving route planning and network efficiency, this approach prioritises human-oriented understanding of service accessibility by analysing how and why people choose specific bus stops.
By doing so, it aims to provide personalised, data-driven suggestions for passengers to help them make more informed travel decisions and improve their public transport experience.