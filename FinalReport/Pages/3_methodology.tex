\section{Methodology}
\label{sec:methodology}

\subsection{Data Source}
\label{subsec:data-source}

This study uses the hourly ridership origin--destination (OD) records from the Bay Area Rapid Transit (BART) system in California, United States~\cite{bart}\@.
Each record represents the number of passengers travelling from a \textit{source} station to a \textit{destination} station within a specific hour.
The dataset is provided in a comma-separated values (CSV) format with the following fields:

\vspace{0.25 cm}
\begin{tabular}{ c c c c }
    \hline
    \textbf{Variable} & \textbf{Description} & \textbf{Value}    \\
    \hline
    date              & service date         & YYYY-MM-DD        \\
    hour              & hour of day          & integer (0 -- 23) \\
    source            & source station       & 4-letter code     \\
    destination       & destination station  & 4-letter code     \\
    passengers        & number of passengers & integer \ldots    \\
\end{tabular}\label{tab:table}
\vspace{0.5 cm}

The station reference table reports the selected attributes used to represent each \textit{stop} in the analysis.
It includes a unique station identifier, descriptive naming fields, and geographic coordinates, which together support consistent station matching across datasets and enable spatial computations.

\vspace{0.25 cm}
\begin{tabular}{ c c c c }
    \hline
    \textbf{Variable} & \textbf{Description}  & \textbf{Value}  \\
    \hline
    code              & station code (unique) & 2-letter uid    \\
    name              & station name          & up to 100 chars \\
    abbreviation      & station abbreviation  & 4-letter code   \\
    latitude          & station latitude      & float           \\
    longitude         & station longitude     & float           \\
\end{tabular}\label{tab:stations-table}

\subsection{Data Preparation}
\label{subsec:data-preparation}

The data preparation methodology consists of three principal stages that transform the raw origin–destination (OD) records into an analysable format.

\begin{itemize}
    \item \textbf{Filtering and Cleaning}: The dataset is first cleaned by removing invalid entries, such as records where the source and destination stops are identical.
    \item \textbf{Stop-Level Aggregation}: Records are grouped by source stop to compute stop-level indicators, including total boardings and destination-choice dispersion (effective destination diversity).
    \item \textbf{Day-Level Aggregation}: The stop-level indicators are aggregated by date to facilitate day-level and time-series analyses.
\end{itemize}

\subsection{Attractiveness Scoring Framework}
\label{subsec:as-framework}

For each stop \(S_i\), the \textbf{Attractiveness Score (AS)} is computed as a weighted aggregation of \textbf{three} metrics derived from the aggregated OD records:
\[ AS_i = w_1 \cdot \textit{Board}_i + w_2 \cdot \textit{EffDst}_i + w_3 \cdot \textit{Access}_i \]

Where:
\begin{itemize}
    \item {
        \textit{Board}\(_i\): Number of boardings at the stop \(S_i\), normalised to \([0,1]\):
        \begin{gather*}
            B_i = \sum_{\text{source}=S_i} passengers \\
            Board_i = \frac{B_i - \min(B)}{\max(B) - \min(B)}
        \end{gather*}
    }
    \item {
        \textit{EffDst}\(_i\): The \textbf{effective destination diversity}, which shows how widely passengers distribute their destination choices (i.e.\ many meaningful options rather than rarely used destinations).
        First, define aggregated OD flow \(F_{ij} = \sum passengers\) for trips from \(S_i\) to \(S_j\), and the destination choice probability
        \begin{gather*}
            p_{ij}=\frac{F_{ij}}{\sum_{j} F_{ij}} \quad \textrm{for } \sum_{j}F_{ij}>0
        \end{gather*}
        Then compute Shannon entropy \(H_i\)~\cite{entropy} and its ``effective number of destinations'' (Hill number of order 1):
        \begin{gather*}
            H_i = -\sum_{j} p_{ij}\ln(p_{ij}), \qquad
            D_i = \exp(H_i)
        \end{gather*}
        Finally, normalise to \([0,1]\):
        \begin{gather*}
            \textit{EffDst}_i = \frac{D_i - \min(D)}{\max(D) - \min(D)}
        \end{gather*}
    }
    \item {
        \textit{Access}\(_i\): An \textbf{attraction-weighted option index} that values access to highly demanded destinations.
        Let destination attractiveness be the inbound volume:
        \begin{gather*}
            A_j = \sum_{\text{destination}=S_j} passengers
        \end{gather*}
        Let \(dist_{ij}\) be the geographic distance between stops \(S_i\) and \(S_j\) (computed from stop coordinates), and let the distance-decay function be:
        \begin{gather*}
            f(dist_{ij}) = \frac{1}{1 + dist_{ij}}
        \end{gather*}
        Then:
        \begin{gather*}
            Acc_i = \sum_{j \neq i} A_j \cdot f(dist_{ij}), \\
            Access_i = \frac{Acc_i - \min(Acc)}{\max(Acc) - \min(Acc)}
        \end{gather*}
    }
    \item \(w_1,w_2,w_3:\) Weight coefficients to be adjusted.
\end{itemize}

\textbf{Note:} \texttt{TransferFrequency} is not considered in this setting because the OD-hour aggregated format does not provide enough information to reliably identify individual transfer events or sequential legs of a single passenger journey.

\subsection{Django Web App Implementation}
\label{subsec:django-web-app}

The interactive environment to display results is built with the Django framework (Python 3.13), following a Model-View-Template (MVT) pattern to manage large-scale transit datasets.
The source code is publicly available on GitHub~\cite{apothemia}.

\subsection{Frontend and Interactive Visualisation}
\label{subsec:frontend}

The presentation layer is a single-page application using \textbf{Leaflet.js} to achieve an interactive map display.

\begin{itemize}[noitemsep]
    \item \textbf{Geospatial Visualisation}: Stops are rendered as interactive markers on top of their corresponding coordinates on the map.
    \item \textbf{Dynamic Parameter Selection}: The interface features several input fields, allowing users to choose the database table holding the records, start and end dates to filter records within an interval, and a ``Weight Splitter'' component to assign coefficients for scoring metrics.
    \item \textbf{Client-Side State Management}: Upon receiving an API response, the system updates the UI with the relevant data received.
\end{itemize}

\subsection{Backend and API Endpoints}
\label{subsec:backend}

The data is stored in a relational database schema of two primary entities: \textit{`Stations'}, containing geographic metadata (coordinates and unique identifiers) of the stops, and \textit{`YearlyUsage'}, containing hourly origin-destination (OD) passenger records.

A custom management script (`load\_csv.py') handles the insertion of high-volume CSV data into relevant databases with transactional batch processing.

The system uses a REST API that serves as the bridge between the database and the visualisation layer.
The key endpoints include:

\begin{itemize}
    \item \textbf{``GET /api/stations/''}: Returns geospatial metadata of stations in JSON format for marker placement.
    \item \textbf{``GET /api/data/''}: Accepts parameters (`model', `start\zwsp{}\_date', `end\zwsp{}\_date') to fetch the data from the given model to return records between the start and end dates.
    \item \textbf{``GET /api/station-scores/''}: Accepts parameters (`start\zwsp{}\_date', `end\zwsp{}\_date') and user-defined weights ($w_1, w_2, w_3$) to return normalised attractiveness metric scores.
\end{itemize}
