\section{Discussion}
\label{sec:discussion}

This section discusses the strengths and limitations of the Attractiveness Scoring (AS) framework, reflects on the challenges encountered, and mentions how alternative choices might influence the outcomes.

\subsection{Strengths and Weaknesses of the Proposed Approach}
\label{subsec:strengths-and-weaknesses-of-the-proposed-approach}

One of the primary strengths of the proposed framework is its complete reliance on observed behaviours.
By using large-scale OD ridership data, the method remains data-driven and reproducible.
Additionally, the modular design of the scoring components provides transparency on how of each scoring metric contributes to the attractiveness score.

However, one key weakness of the AS framework is that there is no direct way of knowing the true attractiveness scores.
Instead, attractiveness is inferred through a weighted combination of multiple behavioural indicators.
For this reason, the choice of weights may introduce an element of subjectivity that can influence stop rankings, unless a more objective method for determining them is provided.

\subsection{Challenges Encountered and Mitigation Strategies}
\label{subsec:challenges}

The most significant challenge was the data availability, as most public transport systems do not make their automated fare collection or smart card transaction data publicly accessible.
As a result, each `station' was considered as `stop' and the stop-level analysis was constrained to station-level for this study.

Another challenge was handling the scale of the dataset, as year-long hourly OD records can be computationally expensive to process.
This was addressed by limiting the analysis window for score calculation to a maximum of seven days, however, this limit is not essential to the framework and can be increased in systems with greater computational capacity.

\subsection{Impact of Alternative Methods}
\label{subsec:alternative-methods}

Alternative methodological choices could lead to different interpretations of attractiveness.

For example, clustering-based approaches may group stops with similar usage profiles without producing an actual ranking.
Similarly, machine learning models could be used to predict attractiveness from external features, but it would also reduce the transparency and interpretability of those features.

Compared to these alternatives, the proposed AS framework prioritises explainability at the expense of predictive performance and relational complexity.