\section{Literature Survey}
\label{sec:literature-survey}


Earlier studies on public transport optimisation~\cite{nasibov2016fuzzypref,bozyigit2018transfer,nasiboglu2022dijkstra} emphasised algorithmic efficiency in route planning using Dijkstra-based methods, fuzzy accessibility measures, or transfer-minimisation strategies.
While these approaches were effective in improving network performance, they largely treated passengers as static objects rather than active decision-makers.
Most existing studies therefore focused on route-level optimisation or network-wide accessibility, overlooking how individual stops influence passengers' travel choices.

More recent work, such as Luo et al.~\cite{luo2025flow}, introduced joint passenger flow prediction models to understand multimodal travel interactions.
This study showed the potential of using large-scale Automatic Fare Collection (AFC) data for uncovering hidden mobility patterns.

However, the literature remains limited when it comes to analysing bus stop attractiveness from a passenger perspective.
This study addresses that gap by evaluating the passenger preferences based on smart card data, using measurable factors such as boarding frequency, temporal patterns, and inter-line connectivity.
Ultimately, the focus shifts from \textit{“How to make a route more optimal?”} to \textit{“Which stop suits a passenger’s needs the most?”}, offering a user-oriented perspective that supports personalised travel recommendations.