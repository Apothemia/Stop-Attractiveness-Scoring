\section{Literature Survey}
\label{sec:literature-survey}


Earlier studies on public transport optimisation~\cite{bozyigit2018transfer} emphasised algorithmic efficiency in route planning through Dijkstra-based methods and transfer-minimisation strategies.
While these approaches were effective in improving network performance, they largely treated passengers as static objects rather than active decision-makers.
Most existing studies therefore focused on route-level optimisation or network-wide accessibility, overlooking how individual stops influence passengers' travel choices.

More recent work, such as Luo et al.~\cite{luo2025flow}, introduced joint passenger flow prediction models to understand multimodal travel interactions.
This study showed the potential of using large-scale Automatic Fare Collection (AFC) data for uncovering hidden mobility patterns.

However, the literature remains limited when it comes to analysing stop attractiveness from a passenger perspective.
This study fills this research gap by analysing passenger preferences through ridership data, focusing on three key metrics: boarding frequency, effective destination diversity, and attraction-weighted accessibility scores.

These metrics provide a quantitative framework for understanding stop attractiveness from a passenger-centric perspective.
Ultimately, the focus shifts from \textit{"How to make a route more optimal?"} to \textit{"What makes each stop attractive to passengers?"}, offering transparency into the decision-making factors that shape public transit usage.
