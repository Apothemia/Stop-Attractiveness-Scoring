\section{Conclusion}
\label{sec:conclusion}

This study was done to analyse public transport usage from the perspective of passengers on stop-level attractiveness rather than network-wide optimisation.
Using large-scale hourly origin--destination ridership data from the San Francisco Bay Area Rapid Transit (BART) system, an Attractiveness Scoring (AS) framework was developed to determine how appealing individual stops are to passengers based on the observed transit data.

\subsection{General Evaluation of the Project}
\label{subsec:general-evaluation}

Overall, the project successfully achieved its primary objective.
The framework was able to transform raw public transportation data into meaningful stop-level indicators.
By combining boarding demand, effective destination diversity, and accessibility-related measures into a single score, the system provides a data-driven way to compare stops.

\subsection{Main Contributions and Achievements}
\label{subsec:contributions}

The main contribution of this work lies in shifting the analytical focus from system optimisation to passenger choice and stop attractiveness.
Unlike traditional studies that prioritise timetable efficiency or route planning, this research introduces a composite attractiveness score based purely on passenger behaviour.

Key achievements include:

\begin{itemize}
    \item The formulation of an Attractiveness Scoring framework which merges multiple behavioural indicators into a single metric.
    \item A data preparation pipeline for handling large-scale public transit datasets.
    \item Providing insights into how transit stops differ in volume of use, diversity, and reach of destinations they serve.
\end{itemize}

\subsection{Suggestions for Future Work}
\label{subsec:future-work}

Several improvements could further enhance this research.
First, the weighting scheme of the AS framework could be optimised or learned automatically using machine learning instead of being user-defined.

Then, having additional contextual data such as transfer-trip frequency, socioeconomic indicators, residential density, or intermodal connectivity could improve the explanatory power.

Finally, validating the attractiveness scores against survey data or passenger satisfaction metrics would significantly improve the interpretation of the results.

\subsection{Real-World Applicability}
\label{subsec:applicability}

The proposed system has practical relevance for both authorities and the public transport users.
Attractiveness scores can support decisions for infrastructural upgrades, service prioritisation, and targeted interventions to encourage public transport usage.

From a passenger-oriented perspective, the framework can be integrated into mobile apps like ``Google Maps'' or ``İzmirimKart'' to help users have a better understanding of public transportation usage.

The approach is also transferable to other cities and transit systems where ridership data is available, making it a flexible tool for public transport analysis.